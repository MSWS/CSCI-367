\documentclass[12pt]{article}
\usepackage{geometry}
\usepackage{fancyhdr}
\usepackage{amsmath}
\usepackage{amssymb}
\usepackage{enumitem}
\usepackage{listings}
\usepackage{xcolor}

\usepackage{libertine}
\usepackage{libertinust1math}
\usepackage[T1]{fontenc}

% Page geometry and margins
\geometry{a4paper, top=2.5cm, bottom=2.5cm, left=2cm, right=2cm}

% Header and Footer settings
\pagestyle{fancy}
\fancyhf{}
\rhead{CSCI 367}
\lhead{Fall 2024}
\rfoot{Page \thepage}

% Listings settings for monospaced font
\definecolor{codegreen}{rgb}{0,0.6,0}
\definecolor{codegray}{rgb}{0.5,0.5,0.5}
\definecolor{codepurple}{rgb}{0.58,0,0.82}
\definecolor{backcolour}{rgb}{0.95,0.95,0.92}

\lstdefinestyle{mystyle}{
    backgroundcolor=\color{backcolour},   
    commentstyle=\color{codegreen},
    keywordstyle=\color{magenta},
    numberstyle=\tiny\color{codegray},
    stringstyle=\color{codepurple},
    basicstyle=\ttfamily\footnotesize,
    breakatwhitespace=false,         
    breaklines=true,                 
    captionpos=b,                    
    keepspaces=true,                 
    numbers=left,                    
    numbersep=5pt,                  
    showspaces=false,                
    showstringspaces=false,
    showtabs=false,                  
    tabsize=2
}

\lstset{style=mystyle}
\author{Isaac Boaz}

\begin{document}
\maketitle

\begin{center}
    \Large \textbf{Lab 1: Socket API}
\end{center}

\vspace{1cm}

For the questions below, you will need to start with and then modify the \texttt{demo\_client.c} and \texttt{demo\_server.c} files linked in the assignment posting.

Note: The term `localhost` refers to the loopback network interface of the machine you are working on, and the IP address `127.0.0.1` is often used to refer to it. When running the client, you can use either `localhost` or `127.0.0.1` as the server address to test communication with a server running on the same machine.

Some additional hints/instructions:
\begin{itemize}
    \item Your answers can be a combination of written explanations and command line output. Please use a monospaced font for any command line output to clearly distinguish it.
    \item Sometimes, the answer to “what happens if... ” could simply be “it breaks, showing the error message ....” (Don't spend too much time trying to force something to work that will not work.)
    \item Hint: For some of these questions, it may be useful to have one process pause while the other performs an action; consider using the \texttt{sleep} command.
\end{itemize}

\vspace{1cm}

\begin{enumerate}[label=Q\arabic*:]
    \item (2 pts) \textbf{Port Questions}
    \begin{enumerate}[label=(\alph*)]
        \item What happens when you run \texttt{demo\_server} using a port in the range of 1 through 1023? Explain why this occurs.
        
        \item What happens when you use a port in the range of 1024 through 65535? Why?
        \item \textit{Optional challenge question (time permitting):} What local port number is \texttt{sd2} (in the server) bound to? (Hint: Use \texttt{getsockname}). Is this what you expected?
    \end{enumerate}

    \item (2 pts) \textbf{recv Questions}
    \begin{enumerate}[label=(\alph*)]
        \item What does \texttt{recv} return when there is still data in the OS buffer, but the other side has closed the socket? To find out, modify your server to send a large amount of data (greater than the client buffer size) and then close the connection immediately. In the client code, add a \texttt{sleep} before calling \texttt{recv}. What do you observe when \texttt{recv} is called under these conditions?

        \item What does \texttt{recv} return when the OS buffer is empty and the other side has closed the socket?
    \end{enumerate}
    
    \item (2 pts) \textbf{listen Questions}
    \begin{enumerate}[label=(\alph*)]
        \item What happens if you comment out the call to \texttt{listen} in \texttt{demo\_server.c}, then compile and run?
        \item If you set the queue size for \texttt{listen} to a small value (e.g., 2), does it actually limit the queue length to 2? How can you test this?
    \end{enumerate}

    \item (2 pts) \textbf{send Questions}
    \begin{enumerate}[label=(\alph*)]
        \item What happens when you change your server’s \texttt{send} call to pass the address of a \texttt{uint32\_t} variable instead of \texttt{buf} and use \texttt{sizeof(uint32\_t)} instead of \texttt{strlen(buf)}- without modifying the client’s \texttt{recv} call?
        \item What happens if you make the above change to \texttt{send}, and also modify your client to perform a single \texttt{recv} call that receives into a \texttt{uint32\_t} variable of the appropriate size? Is the variable successfully transmitted without using any character arrays?
        \item What are \texttt{htonl} and \texttt{ntohl}? What purpose do they serve in socket programming?
        \item When would you need to use \texttt{htonl} and \texttt{ntohl} while transmitting \texttt{uint32\_t} variables (i.e., before or after which actions)?
        \item Modify your client’s \texttt{recv} call to use the \texttt{MSG\_WAITALL} flag, and remove any repeated calls to \texttt{recv}. In the server, change the \texttt{send} function to send partial messages with a delay between each chunk. Test how the client receives data with and without the \texttt{MSG\_WAITALL} flag. What differences do you notice?
    \end{enumerate}

    \item (2 pts) \textbf{Socket Address Structure Questions}
    \begin{enumerate}[label=(\alph*)]
        \item What are the fields of the \texttt{sockaddr\_in} structure? Describe their types, names, and purposes.
        \item What are the fields of the \texttt{in\_addr} structure?
        \item What does the special value \texttt{INADDR\_ANY} mean, and when would you use it?
    \end{enumerate}

    \item (2 pts) \textbf{Address Conversion Questions}
    \begin{enumerate}[label=(\alph*)]
        \item Use \texttt{inet\_aton} to convert a decimal-dotted string IP address to an \texttt{in\_addr}.
        \item What does \texttt{inet\_aton} return when the IP address is valid?
        \item What does \texttt{inet\_aton} return when the IP address isn't valid (e.g., "905.0.0.1")?
        \item Modify your client code (\texttt{demo\_client.c}) to convert the \texttt{h\_addr} field from \texttt{gethostbyname} into a decimal-dotted string format. First, declare a character array of size \texttt{INET\_ADDRSTRLEN} to store the converted IP address. Then, after calling \texttt{gethostbyname}, use \texttt{inet\_ntop(AF\_INET, ptrh->h\_addr, buffer, INET\_ADDRSTRLEN)} to convert the binary IP address to its string representation. Finally, add a \texttt{printf} statement to print the resulting IP address string before establishing the socket connection.
        \item What is printed when you pass host names like "\texttt{google.com}" or "\texttt{linux.cs.wwu.edu}"? Does the output make sense?
        \item What happens when you pass an invalid host name like "\texttt{cf416-99.cs.wwu.edu}"? Why does this happen?
    \end{enumerate}
    
\end{enumerate}

\end{document}