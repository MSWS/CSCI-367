\documentclass[12pt]{article}
\usepackage{geometry}
\usepackage{fancyhdr}
\usepackage{amsmath, amssymb}
\usepackage{enumitem}
\usepackage{xcolor}
\usepackage{hyperref}

% Page geometry and margins
\geometry{a4paper, top=2.5cm, bottom=2.5cm, left=2cm, right=2cm}

% Header and Footer settings
\pagestyle{fancy}
\fancyhf{}
\newcommand{\TheAuthor}{Isaac Boaz and The Philosopher's Stone} % As given in documentation of **fancyhdr**
\newcommand{\Author}[1]{\renewcommand{\TheAuthor}{#1}}
\fancyhead[CO]{\slshape \leftmark}
\fancyhead[C]{\slshape \TheAuthor}
\fancyfoot[C]{\thepage}
\rhead{CSCI 367}
\lhead{Fall 2024}
\rfoot{Page \thepage}

\begin{document}

\begin{center}
    \Large \textbf{Wireshark Lab: UDP and TCP}
\end{center}

\vspace{1cm}

In this lab, you will explore the UDP and TCP transport protocols using packet traces provided in a zip file. Use Wireshark to examine the traces and answer the questions below. Whenever possible, provide annotated printouts of relevant packets.

\section*{UDP Analysis}

\begin{enumerate}[label=Q\arabic*:]
    \item Select one UDP packet from \texttt{http-ethereal-trace-5} (which
          contains some UDP packets carrying SNMP messages). Determine the length (in
          bytes) of each field in the UDP header by examining the details in
          Wireshark. List each field and its length.

          \begin{tabular}{cc}
              Source Port      & 2 bytes \\
              Destination Port & 2 bytes \\
              Length           & 2 bytes \\
              Checksum         & 2 bytes \\
          \end{tabular}

    \item Examine the value in the "Length" field of the UDP header. What does
          this length represent? Verify your answer by comparing it to the combined
          lengths of the header and payload.

          The length field in the UDP header represents the total length of the UDP
          datagram, including the header and payload. This is confirmed by adding the
          lengths of the UDP header fields to the length of the payload to get the
          total length of the UDP datagram.

    \item What is the maximum number of bytes that can be included in a UDP
          payload? \textit{(Hint: Use the field lengths identified in Question Q2 to
              help determine this.)}

          Since the length field in the UDP header is 2 bytes long, the maximum
          value that can be represented is $2^{16} - 1 = 65535$ bytes.
          Subtracting the 8-byte header length gives a maximum payload size of
          $65535 - 8 = 65527$ bytes.

    \item What is the highest possible source port number for UDP?
          \textit{(Hint: Refer to your answer from Q2 for the relevant field length.)}

          The source port field in the UDP header is 2 bytes long, so the highest possible source port number is $2^{16} - 1 = 65535$.

    \item Find the protocol number for UDP in the IP header containing the UDP
          segment. Provide the protocol number in both hexadecimal and decimal
          formats.

          \begin{equation*}
              \text{Protocol Number } = 11_{16} = 17_{10}
          \end{equation*}

    \item Find a pair of UDP packets where one is a reply to the other. Describe the relationship between the source and destination port numbers in these packets.
          The source and destination ports swap between the two packets.
\end{enumerate}

\section*{TCP Analysis}

In this section, you will analyze a TCP transfer recorded in \texttt{tcp-ethereal-trace-1}. This trace captures the upload of a 150KB file from a client to a server, demonstrating TCP sequence numbers, acknowledgments, congestion control, and flow control.

\begin{enumerate}[label=Q\arabic*:, resume]
    \item What is the IP address and TCP port number used by the client to
          transfer the file to the server?

          The client's IP address is \texttt{192.168.1.102}, and uses port \texttt{1161} for the transfer.

    \item Identify the IP address of the server and the port number it uses for
          sending and receiving TCP segments.

          The server's IP address is \texttt{128.119.245.12}, and it uses port
          \texttt{80} for both sending and receiving segments.

    \item Locate the SYN segment initiating the connection to the server. What
          is the sequence number of this segment? What field indicates this segment as
          a SYN?

          The sequence number of the SYN segment is \texttt{232129012}. The SYN flag in the
          TCP header indicates that this segment is a SYN (second to last bit in
          the flags field).

    \item Find the SYNACK segment sent by the server. Record its sequence
          number, the Acknowledgement field value, and explain how the server
          calculated this acknowledgment.

          The sequence number of the SYNACK segment is \texttt{883061785}. The Acknowledgement
          field value is \texttt{232129013}. The server calculated this acknowledgment
          by adding 1 to the sequence number of the SYN segment received from the client.

    \item What is the sequence number of the TCP segment containing the HTTP
          POST command? \textit{(Hint: Look in the DATA field for "POST".)}

          \texttt{232293053}

    \item Starting with the HTTP POST, record the sequence numbers of the first
          six TCP segments, the time each segment was sent, and the time each ACK was
          received. For each segment, calculate the Round-Trip Time (RTT), which is
          the time difference between when a segment is sent and when its ACK is
          received. What is the average RTT for the first 3 segments sent by the
          client?

          \begin{tabular}{|c|c|c|c|c|}
              \hline
              Segment & Sequence Number & Sent Time  & RTT         & Length   \\
              \hline
              1       & 232293053       & 1093095865 & 0.158489000 & 20 + 50  \\
              2       & 883061786       & 1093095865 & N/A         & 20       \\
              3       & 883061786       & 1093095866 & N/A         & 20       \\
              4       & 883061786       & 1093095866 & N/A         & 20       \\
              5       & 883061786       & 1093095866 & N/A         & 20 + 730 \\
              6       & 164091          & 1093095866 & N/A         & 20       \\
              \hline
          \end{tabular}

    \item What is the length (in bytes) of each of the first six TCP segments?
          Refer above.

    \item Examine the advertised buffer space at the receiver by inspecting the
          "Window Size" field in the TCP header of each ACK packet sent by the
          receiver. Record the smallest value found in the "Window Size" field across
          all packets. This minimum value represents the least amount of buffer space
          available at any point during the connection. Based on your observations,
          explain whether there were times when a lack of buffer space restricted the
          sender’s ability to transmit data. \textit{(Hint: Sort the packets by the
              source IP address.)}

          The smallest window size observed was 5840 bytes, and it increased
          throughout the connection. The sender's ability to transmit data was not
          restricted by the receiver's buffer space, as the window size was always
          sufficient to accommodate the data being sent.

    \item Are any TCP segments retransmitted in this trace? If so, what are the
          packet numbers? If not, how did you determine that there were no
          retransmissions?

          There are no retransmissions in this trace. This was determined by
          examining the sequence numbers of the transmitted segments and
          comparing them to the acknowledgment numbers received from the
          receiver. All segments were acknowledged without any retransmissions.

    \item How much data does the receiver typically acknowledge in each ACK?
          Identify cases where the receiver acknowledges every other segment.

          The receiver seems to be acknowledging about 3000 bytes of data in each ACK.
          The receiver almost always acknowledges every other segment.

\end{enumerate}

\end{document}
