\documentclass{article}

\usepackage{amsmath, amssymb}

\usepackage{multicol}
\usepackage{libertine}
\usepackage{libertinust1math}
\usepackage[T1]{fontenc}

\usepackage[a4paper, margin=1in]{geometry}

\title{Simplex, Duplex, Multiplex}
\author{Logan Sizemore}
\setcounter{secnumdepth}{5}
\setcounter{tocdepth}{5}

\begin{document}

\maketitle

How do we share a transmission medium?
\begin{description}
    \item[Simplex] One-way communication (one-way road) \\
          \textbf{Example:} Optical fiber
    \item[Duplex] Two-way communication (two-way road); transmit and receive simultaneously \\
          \textbf{Example:} `Full' optical fiber cable, with some fibers for transmitting and some for receiving
    \item[Half-Duplex] Two-way communication (two-way road); transmit and receive, but not simultaneously \\
          \textbf{Example:} Walkie-talkie, construction zone with alternating traffic
    \item[Multiplexing] the process of sharing a tranmission medium among multiple devices / signals. \\
          Take several distinct signals and interleave them in a way that can be be separated at the receiving end.
\end{description}

\subsection*{Multiplexing}
\subsubsection*{Frequency Division Multiplexing (FDM)}
\begin{itemize}
    \item Divides medium into different frequency bits
    \item Each signal is sent simultaneously but on a different frequency band
          \begin{itemize}
              \item FM Radio
              \item Cable TV
          \end{itemize}
\end{itemize}

\subsubsection*{Time Division Multiplexing (TDM)}
\begin{itemize}
    \item Devices take turns using the medium in a round-robin manner.
    \item Multiplexes signals at the baseband level by assigning different time slots to each signal.
    \item Is `baseband' if the signal uses the entire bandwidth of the medium.
\end{itemize}

\subsubsection*{Statistical Time Division Multiplexing}
A more dynamic version of TDM where time slots are based on needs rather than
fixed turns.
Efficient use of bandwidth by allocating time only when devices need it.
FDM and TDM are not mutually exclusive and can be combined.

\newpage
\section*{Ethernet and Network Technologies}
We know how data is sent across a transmission medium.
How is data transferred over a local network?

\subsection*{WAN and LAN}
WAN is strictly in Layer 2.
\begin{multicols}{2}
    \begin{description}
        \item[LAN] Small geographic span
              \begin{itemize}
                  \item 10s to 1000s of meters
                  \item Used in offices and homes
                  \item Modest bandwidth
                  \item Low latency
                  \item ethernet / WiFi
              \end{itemize}
    \end{description}
    \begin{description}
        \item[WAN] Large geographic span
              \begin{itemize}
                  \item state, country, globe
                  \item Connects networks via routers
                  \item Huge bandwidth
                  \item High latency
                  \item Optical fiber / satellite
              \end{itemize}
    \end{description}
\end{multicols}

\subsection*{Principles of Ethernet}
Designed as a simple way to connect multiple devices in a local network.
\begin{enumerate}
    \item All devices connect to a \emph{common communication channel}, sharing
          access to transmit data.
    \item Any message sent is \emph{broadcasted to all hosts} on the network,
          and each host determines if the message is relevant to them.
    \item Data transmission is \emph{unreliable} - there is no guarantee of
          delivery, and no acknowledgment of receipt.
    \item There is no \emph{central authority} managing access to the channel.
          Hosts \emph{negotiate access} amongst themselves.
\end{enumerate}

\subsubsection*{Thicknet (10Base5)}
\begin{itemize}
    \item Shared coaxial cable bus
    \item Attachment Unit Interface (AUI) cable run to a transceiver, which
          connects to the computer's network card.
    \item 10Base5: 10 Mbps, Baseband, max length 500 meters.
    \item Thich rigid cables (hard to install).
\end{itemize}

\subsubsection*{Thinnet (10Base2)}
\begin{itemize}
    \item Replaced thicknet with thinner coaxial cables
    \item Shared bus runs from computer to computer using BNC-T connectors
    \item More flexible and easier to work with.
    \item Chained directly from one computer to the next.
    \item 10Base2: 10 Mbps, Baseband, max length 200 meters.
\end{itemize}

\subsubsection*{Hub-Based Twisted Pair (10BaseT / 100BaseTX)}
\begin{itemize}
    \item Uses twisted pair wires (CAT5, 6, 8, etc.)
    \item All devices connect to a central hub (`bus in a box')
    \item No taps, instead connects directly from network card to hub
    \item 10BaseT: 10 Mbps, Baseband, max length 100 meters.
    \item Speeds increased to 10 / 100 / 1000 Mbps
\end{itemize}

\subsubsection*{Switch-Based Twisted Pair (1000BaseTX)}
\begin{itemize}
    \item Switches replace hubs for more efficient data transmission
    \item Each device has dedicated ethernet cable running to switch
    \item Switch enables all devices to connect and communicate efficiently
\end{itemize}

\subsection*{Ethernet Hardware Addressing}
On a shared bus network, every device (host) receives all messages.
Thus, each host must determine `is this frame for me?'
To achieve this, every frame contains:
\begin{itemize}
    \item Source HW address
    \item Destination HW address
\end{itemize}

\subsubsection*{MAC Address}
\emph{Media Access Control} address used for frame delivery.
Structure
\begin{itemize}
    \item 48 bits (6 bytes)
    \item Written in hexadecimal format (e.g., B8:27:EB:1E:2D:4F)
    \item Colons separate octets
\end{itemize}

\paragraph*{Assignment of MAC Addresses}
Manufacturers buy blocks of MAC addresses from IEEE.

\begin{equation*}
    \mathtt{\underbrace{B8}_{\text{asdf}}:CA:3A:B9:F3:D2}
\end{equation*}

Can temporarily change (spoof)
\begin{itemize}
    \item Debugging
    \item Malicious activities
\end{itemize}

\begin{description}
    \item[Unicast] Specifies a single device.
    \item[Broadcast] All devices on the network. \\
          All 1s in binary, FF:FF:FF:FF:FF:FF in hexadecimal.
    \item[Multicast] Sent to a specific group of devices.
\end{description}

\subsubsection*{Promiscuous Mode}
\begin{itemize}
    \item Allows a network card to accept all frames, regardless of destination.
    \item Commonly used for network debugging or Wi-Fi monitoring.
\end{itemize}

\subsection*{Ethernet Frames}
\emph{Link Layer} exists to exchange messages across a \textit{physical
    network}.

Messages are frames...
\end{document}