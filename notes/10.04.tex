\documentclass{article}

\usepackage{amsmath, amssymb}

\usepackage{multicol}
\usepackage{libertine}
\usepackage{libertinust1math}
\usepackage[T1]{fontenc}

\usepackage[a4paper, margin=1in]{geometry}

\title{Transmission Media}
\author{Logan Sizemore}
\setcounter{secnumdepth}{5}
\setcounter{tocdepth}{5}

\begin{document}

\maketitle

\paragraph*{Guided vs. Unguided Media}
\begin{tabular}{cc}
    Guided                              & Unguided                   \\
    Propogates through a physical cable & Propogates through air     \\
    Twisted pair, coaxial, fiber optic  & Radio, microwave, infrared \\
\end{tabular}

\begin{itemize}
    \item Electricity \begin{itemize}
              \item Twisted pair cable
              \item Caoxial Cable
          \end{itemize}
    \item Light (fiber optic)
    \item Radio Signals (Wi-Fi)
\end{itemize}

\paragraph*{Guided Electrical Transmission}
Signal is formed by modulating voltage through the medium.
Requires two wires to form a complete circuit.

\begin{multicols}{2}
    Main issue is \emph{electromagnetic radiation}.
    Can be mitigated in two ways:
    \begin{enumerate}
        \item Twisted Cables
        \item Shielding
    \end{enumerate}
    \begin{tabular}{c|c}
        Advantages      & Disadvantages               \\ \hline
        Cost-effective  & Susceptible to interference \\
        Easy to install & Limited length (< 100 m)    \\
        Durable         &                             \\
        High-speed      &                             \\
    \end{tabular}
\end{multicols}


\subsubsection*{Twisted Pair}
Two wires (usually copper) twist together, one for signal, one for ground.
\emph{Twisting} the wires ensures they are \emph{exposed equally} to radiation,
minimizing the chance of noise.


\subsubsection*{Shielding}
Twisted pair wiring has problems when:
\begin{itemize}
    \item Noise is especially strong / close
    \item Data is transmitted at high frequency
\end{itemize}

Metal shielding that surroudns the signal wire can protect it from interference.

\subsection*{Optical Fiber}
Optical fiber cables are composed of long, thin strands of glass encased in
plastic.
Data is transmitted as short, well-defined light pulses.

\begin{quote}
    As light travels through the fiber, it can scatter and broaden.
\end{quote}

On either end there is
\begin{description}
    \item[Light emitter]  (or LED) sends data in the form of light pulses.
    \item[Photosensitive detector] receives and interprets the light pulses.
\end{description}

\subparagraph*{Total Internal Reflection} can contribute to smearing due to some
photons taking longer paths than others.

\begin{tabular}{c|c}
    Advantages       & Disadvantages          \\ \hline
    High speed       & Expensive              \\
    Less attenuation & Difficult to work with \\
    Security         &                        \\
    High Bandwidth   &                        \\
    Repeatability
\end{tabular}

\subsection*{Radio Based Transmission}

A form of unguided transmission.

Commonly used for Wi-Fi, Bluetooth, satelliate, etc.
\begin{itemize}
    \item Low Frequency (LF): 30 kHz - 300 kHz (long-range, maritime signals)
    \item Very High Frequency (VHF): 30 MHz - 300 MHz (FM radio, TV)
    \item Ultra High Frequency (UHF): 300 MHz - 3 GHz (Wi-Fi, Bluetooth)
    \item Microwave: 3 GHz - 300 GHz (satellite, radar)
\end{itemize}

Governed by the \emph{FCC} in the US.

\begin{tabular}{cc}
    Advantages  & Disadvantages               \\ \hline
    Ease of use & Security Risks              \\
    Flexible    & Susceptible to noise        \\
                & Limited bandwidth and range
\end{tabular}
\end{document}