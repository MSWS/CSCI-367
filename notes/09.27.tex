
\documentclass{article}

\usepackage{amsmath, amssymb}

\usepackage{libertine}
\usepackage{libertinust1math}
\usepackage[T1]{fontenc}
\usepackage[a4paper, margin=1in]{geometry}

\title{Protocols and the Layered Model}
\author{Logan Sizemore}

\begin{document}


\maketitle

\section{Protocols}
\subsection{Morse Code}
Invented by Samuel Finley Breese Morse, morse code is an example of a
\emph{protocol} that simplifies communication by standardizing messages into a
known format.

\begin{itemize}
    \item A \emph{protocol} is a set of rules that guide how devices communicate with each
          other.
    \item Defines the format (syntax) and meaning (semantics) of the data being exchanged.
\end{itemize}

\noindent
Expected characteristics of a network
\begin{enumerate}
    \setlength{\itemsep}{0em}
    \item Scalable
    \item Secure
    \item Reliable
\end{enumerate}

\subsection{TCP/IP}
\begin{description}
    \item[Transmission Control Protocol (TCP)] ensures reliable data transfer
          and performs error-checking between devices.
    \item[Internet Protocol (IP)] is responsible for addressing and routing
          between devices on a network or across the internet.
    \item[Internet Architecture Board (IAB)] is responsible for defining the
          standards that make up the internet.
          \begin{itemize}
              \item Coordinated the technical direction of the internet.
              \item Set official policies for its development.
              \item Determined which protocols were essential to the TCP/IP suite.
          \end{itemize}
\end{description}

\subsubsection{IETF and RFCs}
The IAB was reorganized into the \emph{Internet Engineering Task Force (IETF)}.

\subsection{IPv4}
IP Addresses is a 32-bit unsigned integer.
Eg:
\begin{enumerate}
    \setlength{\itemsep}{0em}
    \item \texttt{3,232,235,777}
    \item We convert to \texttt{11000000 11110000 00000000 00000001}
    \item Each is then converted to decimal and separated by a period.
    \item \texttt{192.168.1.1}
\end{enumerate}

\large
\section{Layered Model}
\begin{enumerate}
    \item[5] Application Layer
    \item[4] Transport Layer
          \begin{itemize}
              \item Ensures reliable data between applications across networks.
              \item Uses UDP Datagrams or Segments.
          \end{itemize}
    \item[3] Internetwork Layer
          \begin{itemize}
              \item Responsible for logical addressing (IP), routing, and forwarding
                    data \emph{between networks}.
              \item Determines best path for data to travel.
          \end{itemize}
    \item[2] Link Layer
          \begin{itemize}
              \item Responsible for local network communication.
              \item Controls how data is packaged into frames.
          \end{itemize}
          \begin{itemize}
              \item MAC Address
          \end{itemize}
    \item[1] Physical Layer
          \begin{itemize}
              \item Responsible for the transmission of raw bits from A to B.
              \item Does not care about addresing; whoever is listening will receive the message.
          \end{itemize}
          \begin{itemize}
              \item Wires
          \end{itemize}
\end{enumerate}
\normalsize

\end{document}