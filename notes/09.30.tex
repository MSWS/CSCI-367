\documentclass{article}

\usepackage{amsmath, amssymb}

\usepackage{libertine}
\usepackage{libertinust1math}
\usepackage[T1]{fontenc}

\usepackage[a4paper, margin=1in]{geometry}

\title{Layered Model}
\author{Logan Sizemore}
\setcounter{secnumdepth}{5}
\setcounter{tocdepth}{5}

\begin{document}

\maketitle

\section{Layered Model}
\subsection{Physical Layer}
Responsible for the transmission of raw bits over a physical medium from point A
to B.

\begin{itemize}
    \item Ethernet, bluetooth, wifi, etc.
    \item No addressing required
\end{itemize}


\subsection{Link Layer}
Responsible for local network communication, how data is packaged into frames,
and ensuring error-free data transfer between nodes.

Protocols around Ethernet / Wi-Fi are in this layer.

Typically, MAC addressing is used.

\subsection{Internetork Layern}
Responsible for logical addressing, routing, and forwarding data between
networks.

\subsection{Transport Layer}
Ensures reliably communication between applications across networks.
Uses UDP Datagrams or Segments.
Typically used with ports.

\begin{quote}
    How are we delivering the data to/from the application?
\end{quote}

\subsection{Application Layer}
Provides protocols for end-user applications.
Application layer is like an API for the transport layer.

\subsubsection{Client-Server Model}
\begin{itemize}
    \setlength{\itemsep}{0em}
    \item A client is a host that sends requests to a server to get data.
    \item A server is a host that provides data to clients.
\end{itemize}

Any host on a network could act as either a client or server.

For clients to be able to connect to servers, they need to know its server
address.

\begin{tabular}{cc}
    Servers                          & Clients                              \\
    \hline
    Operate continuously             & Request services                     \\
    Provide services to many clients & Operate intermittently               \\
    At a known location              & Don't need to be at a known location \\
\end{tabular}

\paragraph{Load Balancing} Use multiple machines, with a server forwarding data
to other machines.
\paragraph{Proxy Cache}
\paragraph{Client Cache}

\section{Socket API}
Originally developed for Berkley Unix (1983)
Acts like a 2-way pipe, with two services supported:
\begin{description}
    \item[Stream]  reliably send a sequence of bytes
    \item[Datagram] unreliably send messages
\end{description}
\end{document}