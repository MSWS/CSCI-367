\documentclass{article}

\usepackage{amsmath, amssymb}

\usepackage{multicol}
\usepackage{libertine}
\usepackage{libertinust1math}
\usepackage[T1]{fontenc}

\usepackage[a4paper, margin=1in]{geometry}

\title{Routing and Supernetting}

\begin{document}

\maketitle

\section*{Handling Datagrams}

A datagram can arrive to a host, or a router.

\subsection*{Host}
If datagram is for me, there are three possible reasons:
\begin{description}
    \setlength{\itemsep}{0em}
    \item[Unicast] Meant for the host specifically.
    \item[Broadcast] Meant for everyone.
    \item[Multicast] Meant for a group of hosts.
\end{description}

If it's not for `me', then drop it.

\subsection*{Router}
If datagram is for `me', accept it.
If not for `me', decrement TTL, if it reaches 0, drop it and send ICMP message.
Otherwise, recompute checksum in IP header and then deliver it (either directly
or indirectly).

\begin{description}
    \item[Direct] Router is directly connected to the destination.
    \item[Indirect] Router has to forward the packet to another router.
\end{description}

\section*{Routing}
Each router has a routing table, which maps from Network ID -> Router.

\begin{equation*}
    \left(\underbrace{N\left(I_x\right)}_{\text{network}}, \underbrace{R_2}_{\text{next hop}}\right)
\end{equation*}

\begin{align*}
    X \rightarrow Z & \text{ typically doesn't change} \\
    Z \rightarrow X & \text{ sometimes changes}
\end{align*}

Thus, every router doesn't need to know the entire network, just the next hop.

\section*{Supernetting}
\begin{itemize}
    \item Reduce size of routing tables
    \item Make routing easier
    \item Make `big networks' have a single entry in the routing table
\end{itemize}
\end{document}