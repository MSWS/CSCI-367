
\documentclass{article}

\usepackage{amsmath, amssymb}

\usepackage{multicol}
\usepackage{libertine}
\usepackage{libertinust1math}
\usepackage[T1]{fontenc}

\usepackage[a4paper, margin=1in]{geometry}

\title{Transmission Errors and Correction}
\author{Logan Sizemore}
\setcounter{secnumdepth}{5}
\setcounter{tocdepth}{5}

\begin{document}

\maketitle
\subsection*{Source of transmission errors}
\begin{enumerate}
    \item Noise - Unwanted \emph{random} signals or fluctuations
    \item Interference - Disruption caused by \emph{external signals}
    \item Distortion - \emph{Changes in the shape} or form of the data signal
\end{enumerate}

\subsubsection*{Single Bit Error}
\texttt{1010110} -> \texttt{10\textbf{0}0110}

\subsubsection*{Burst Error}
A group of bits within a block are changed due to a longer-duration disturbance.
Often caused by sustained interferences over a longer period of time.

Bursts start at the first error and end at the last error.

\subsubsection*{Erasure}
A signal received is ambiguous or unclear.
Often caused by distortion or strong interference.

\subsection*{Hamming Distance}
Given two strings of $n$ bits each, the Hamming Distance is defined as the numebr of differences.
\texttt{10110001011} and \texttt{10011010111} have a Hamming Distance of 5.

\subsection*{Error Handling}
If we \emph{know} that there is an error in some data that we received...
\begin{itemize}
    \item Ask sender to send the data again.
    \item Correct error on our side.
\end{itemize}

\subsubsection*{Correction vs. Detection}
\begin{description}
    \item[Detection] Identify whether an error has occurred during data. \\
          Generally simpler with less overhead.
    \item[Correction] Identify and correct the error. \\
          More complex and requires redundancy, but allows for self-correction without retransmission.
\end{description}

\subsubsection*{Redundancy}
One method for error detection is introducing redundancy.

Notationally:
\begin{itemize}
    \item a \emph{dataword} is the original data
    \item a \emph{codeword} is the dataword with redundancy added
\end{itemize}

if a dataword has $k$ bits and $r$ redundant bits are added, the encoding is called an

\begin{equation*}
    (n, k) \text{ encoding scheme, where } n = k + r
\end{equation*}

\begin{tabular}{cl}
    Sender   & Receiver                  \\
    Dataword & Dataword                  \\
    Encoder  & Decoder -> Redundant Bits \\
    Codeword & Discard
\end{tabular}

\subsection*{Parity}
Single Parity Checking (SPC) adds a single \emph{parity bit} to data to ensure an even or odd number of 1s.

If a burst error occurs which changes an even number of bits, the receiver will
incorrectly classify the data as correct.

No error detection is perfect, some errors may go undetected.
The goal becomes:
\begin{enumerate}
    \item Minimize the possibility that one valid codeword can be transformed
          into another valid codeword.
    \item Minimize the number of redundant bits we must add to the message.
\end{enumerate}

\subsubsection*{Hamming Distance and Robustness}
\begin{multicols}{2}
    \begin{tabular}{cc}
        Data & Code \\
        00   & 001  \\
        01   & 010  \\
        10   & 100  \\
        11   & 111  \\
    \end{tabular} \\
    All have a hamming distance between each other of 2.
\end{multicols}

\paragraph*{Error Detection Trade Off}
A larger min distance \((d_{min})\) between codewords is beneficial for error
detection.

If fewere than \(d_{min}\) bits are changed, the error can be detected.

The max number of detectable bit errors is:
\begin{equation*}
    e = d_{min} - 1
\end{equation*}

\subsubsection*{RAC Parity}
Error correction method using 2D parity checks.
Data organized into a grid with parity bits added for each row and column.

\end{document}