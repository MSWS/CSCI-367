\documentclass{article}

\usepackage{amsmath, amssymb}

\usepackage{libertine}
\usepackage{libertinust1math}
\usepackage[T1]{fontenc}

\title{Computer Networks}
\author{Logan Sizemore}

\begin{document}

\maketitle

A computer network is a communications network that allows devices to exchange
information or data.

\subsection{Glossary}

\begin{description}
    \item[Nodes]  are the individual devices or components that connect to a
          network; essentially any active, addressable point in the network.
          \begin{itemize}
              \item Send
              \item Receive
              \item Store
              \item Forward
          \end{itemize}
    \item[Links] are the connections between nodes. They represent the
          medium through which data is transmitted. Can be physical or wireless.
    \item[Hosts] are nodes in a network which run applications.
    \item[Internetwork] is the collection of two or more networks connected via routers.
\end{description}

\noindent
Non-obvious hosts include:
\begin{itemize}
    \item Video-Game Consoles
    \item IoT Devices (Ring, Alex, Fridges, etc.)
    \item Pagers
    \item Card Payment Devices / ATMs
\end{itemize}

\section{Creation}
\subsection{ARPANET}
\subsubsection{TCP/IP Standard}

\subsection{Metcalfe's Law}
\newtheorem{metcalfe}{Metcalfe's Law}

\begin{metcalfe}
    The value of a network grows \(O(n^2)\), where $n$ is the number of nodes in
    a network.
\end{metcalfe}

i.e. it is better for \emph{a} network to have 100 nodes, than 10 networks with
10 nodes.

\end{document}