\documentclass{article}

\usepackage{amsmath, amssymb}

\usepackage{multicol}
\usepackage{libertine}
\usepackage{libertinust1math}
\usepackage[T1]{fontenc}

\usepackage[a4paper, margin=1in]{geometry}

\title{Internet Standards Organization Open Systems Interconnected (ISO OSI) Model}

\begin{document}

\maketitle

\section*{Layers}
\begin{enumerate}
    \item Application
    \item Presentation
    \item Session
    \item Transport
    \item Internetwork
    \item Link
    \item Physical
\end{enumerate}

First three layers are considered the \emph{Application} layer.
The session layer keeps track of the state.
Presentation layer is responsible for image/text formats.

\section*{TCP/IP Reference Model}
Specifically built around the TCP/IP protocol suite.

\subsection*{Layers}
\begin{enumerate}
    \item Application
    \item Transport
    \item Internet
    \item Network Interface
\end{enumerate}

\section*{Hour Glass Model}
Presented as an hour glass, where the top and bottom are the application and
physical layers, respectively. The upper and lower layers are `bigger' (hence
the hour glass) with the middle layer (network) being the smallest.

Above transport is outside of the OS.
Above Link layer is IP address, below is hardware address.
\end{document}