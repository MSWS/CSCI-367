\documentclass[8pt]{extarticle}

\usepackage{lipsum}
\usepackage{enumitem}
\usepackage{multicol}
\usepackage{multirow}
\usepackage{bytefield}

\usepackage[margin=0.2in]{geometry}

\setdescription{itemsep=-3pt}
\setenumerate{itemsep=0pt}
\setitemize{itemsep=0pt}

\begin{document}
\begin{tabular}{|c|c|}
  \hline
  Application  & HTTP, FTP, SMTP \\
  \hline
  Transport    & TCP, UDP        \\
  \hline
  Internetwork & Datagrams       \\
  \hline
  Link         & MAC, Frames     \\
  \hline
  Physical     & Ethernet, WiFi  \\
  \hline
\end{tabular}

\section*{Glossary}
\begin{description}
  \item[MTU]  Maximum Transmission Unit
        \begin{itemize}
          \item Calculated as the smallest MTU of all links in the path.
          \item If a packet is larger than the MTU, it is fragmented.
        \end{itemize}
  \item [ECN] Explicit Congestion Notification
        \begin{itemize}
          \item A bit in the IP header that is set by the router to indicate congestion.
          \item Related to TCP's congestion control.
        \end{itemize}
  \item [Subnetting] Dividing a network into smaller networks.
        \begin{itemize}
          \item Done by borrowing bits from the host portion of the address.
          \item Done to reduce broadcast traffic and improve security.
        \end{itemize}
  \item [Supernetting] Combining multiple networks into a single network.
        \begin{itemize}
          \item Done by borrowing bits from the network portion of the address.
          \item Done to reduce routing table size.
        \end{itemize}
\end{description}

\section*{Physical}
\begin{description}
  \item[CAT] Category (twisted pair) cable for Ethernet.
  \item[Coax] Coaxial cable for Ethernet.
\end{description}

\section*{Link}
\begin{description}
  \item[MAC] Media Access Control
        \begin{itemize}
          \item Unique identifier for a network interface.
          \item 48 bits, often represented as 6 groups of 2 hexadecimal digits.
          \item First 24 bits are the OUI (Organizationally Unique Identifier).
          \item Example: \texttt{00:1A:2B:3C:4D:5E}
        \end{itemize}
  \item[ARP] Address Resolution Protocol
        \begin{itemize}
          \item Who has \texttt{$IP_B$}? Tell \texttt{$IP_A$} at \texttt{$MAC_A$}.
                \begin{enumerate}
                  \item Any listening devices may receive this request and update their
                        local cache of $IP_A$'s $MAC_A$.
                  \item If the receiving machine is not $IP_B$, it will stop processing
                        the request.
                  \item Otherwise, it will send an ARP reply to $MAC_A$. \\
                        Since this reply is \emph{not} broadcasted, no other machines can update their cache.
                \end{enumerate}
          \item Typical cache timeout of 20 minutes.
          \item Gratuitous ARP is used to update caches.
                \begin{enumerate}
                  \item A machine sends an ARP request for its own IP.
                  \item All machines update their cache with the new MAC.
                  \item This is useful for updating caches after a machine has changed its MAC.
                \end{enumerate}
        \end{itemize}
  \item[Switch] A device that forwards packets based on MAC addresses. (Layer 2)
  \item[Hub] A device that broadcasts packets to all connected devices. (Layer 1)
\end{description}

\section*{Link}
\subsection*{Transmission}
\begin{description}
  \item[Direct and Indirect Deliver] Direct is on the same network, indirect is on a different network.
  \item[Unicast, Broadcast, Multicast] Unicast is one-to-one, broadcast is one-to-all, multicast is one-to-many.
  \item[Switching] Forwarding packets based on MAC addresses.
  \item[Spanning Tree Protocol] Prevents loops in a network.
\end{description}

\section*{Internetwork}
\subsection*{IP Datagrams}
\subsubsection*{Fragmentation}
\begin{itemize}
  \item Done by routers when the packet is too large for the next hop.
  \item \texttt{DF} (Don't Fragment) bit in the IP header can be set to prevent fragmentation.
  \item \texttt{MF} (More Fragments) bit is set on all fragments except the last.
  \item \texttt{Fragment Offset} field is used to reassemble the fragments.
  \item \texttt{Identification} field is used to identify the original packet.
  \item \texttt{Total Length} field is the size of the fragment.
  \item \texttt{Header Checksum} is recalculated for each fragment.
\end{itemize}

\section*{Transport}
\begin{multicols}{2}
  \subsection*{UDP}
  \begin{itemize}
    \item Connectionless, unreliable.
    \item Includes psuedo-header in checksum calculation.
  \end{itemize}

  \columnbreak

  \subsection*{TCP}
  \begin{itemize}
    \item Stream Orientation.
    \item Virtual Circuit.
    \item Buffered, unstructured stream.
    \item Full duplex.
  \end{itemize}
\end{multicols}

\section*{Schemas}
\small

\subsubsection*{Ethernet Frame}
\begin{bytefield}{32}
  \bitbox[]{8}{Preamble} & \bitbox[]{6}{Destination Address} &
  \bitbox[]{6}{Source Address} & \bitbox[]{2}{Frame Type} & \bitbox[]{12}{Frame
    Data} & \bitbox[]{4}{CRC} \\
  \bitbox{8}{8 octets} & \bitbox{6}{6 octets} & \bitbox{6}{6 octets} &
  \bitbox{2}{2} & \bitbox{12}{46 - 1500 octets} & \bitbox{4}{4 octets} \\
\end{bytefield}

\subsubsection*{IPv4 Datagram}
\begin{bytefield}{32}
  \bitheader{0, 8, 16, 31} \\
  \bitbox{4}{VERS} & \bitbox{4}{HLEN} & \bitbox{8}{SERVICE TYPE} &
  \bitbox{16}{TOTAL LENGTH} \\
  \bitbox{16}{IDENTIFICATION} & \bitbox{3}{FLAGS} & \bitbox{13}{FRAGMENT OFFSET}
  \\
  \bitbox{8}{TTL} & \bitbox{8}{PROTOCOL} & \bitbox{16}{HEADER CHECKSUM} \\
  \wordbox{1}{SOURCE IP} \\
  \wordbox{1}{DESTINATION IP} \\
  \bitbox{24}{IP OPTIONS (IF ANY)} & \bitbox{8}{PADDING} \\
  \wordbox{1}{DATA \dots} \\
\end{bytefield}

\begin{multicols}{2}
  \subsubsection*{UDP Header}
  \begin{bytefield}{32}
    \bitheader{0, 16, 31} \\
    \bitbox{16}{SOURCE PORT} & \bitbox{16}{DESTINATION PORT} \\
    \bitbox{16}{LENGTH} & \bitbox{16}{CHECKSUM (or 0)} \\
    \wordbox{1}{DATA \dots} \\
  \end{bytefield}

  \columnbreak

  \subsubsection*{UDP Psuedo Header}
  \begin{bytefield}{32}
    \bitheader{0, 8, 16, 31} \\
    \wordbox{1}{SOURCE IP} \\
    \wordbox{1}{DESTINATION IP} \\
    \bitbox{8}{Zero} & \bitbox{8}{PROTOCOL} & \bitbox{16}{UDP LENGTH} \\
  \end{bytefield}
\end{multicols}

\normalsize

\section*{Addressing}

\begin{multicols}{2}
  \subsection*{Classful Addressing}
  \begin{tabular}{c|c|l}
    Class A & \texttt{0XXXXXXX} & /8  \\
    Class B & \texttt{10XXXXXX} & /16 \\
    Class C & \texttt{110XXXXX} & /24 \\
    Class D & \texttt{1110XXXX} & /31 \\
    Class E & \texttt{1111XXXX} & /32 \\
  \end{tabular}

  \begin{description}
    \item[`This' network'] has host bits set to 0.
    \item[Broadcast] has host bits set to 1.
  \end{description}

  \subsection*{ARP}

  \columnbreak

  \begin{tabular}{|c|c|c|c|l}
    \cline{0-3}
    \multicolumn{4}{|c|}{all 0s} & This host                                               \\
    \cline{0-3}
    \multicolumn{5}{c}{}                                                                   \\
    \cline{0-3}
    \multicolumn{2}{|c|}{all 0s} & \multicolumn{2}{|c|}{host}               & Host on this
    network                                                                                \\
    \cline{0-3}
    \multicolumn{5}{c}{}                                                                   \\
    \cline{0-3}
    \multicolumn{4}{|c|}{all 1s} & Limited broadcast (local net)                           \\
    \cline{0-3}
    \multicolumn{5}{c}{}                                                                   \\
    \cline{0-3}
    \multicolumn{2}{|c|}{net}    & \multicolumn{2}{|c|}{all 1s}             & Directed
    broadcast for net                                                                      \\
    \cline{0-3}
    \multicolumn{5}{c}{}                                                                   \\
    \cline{0-3}
    127                          & \multicolumn{3}{|c|}{anything (often 1)} &
    Loopback                                                                               \\
    \cline{0-3}
  \end{tabular}
\end{multicols}

\end{document}